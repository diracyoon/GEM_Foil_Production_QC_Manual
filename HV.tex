\section{고전압을 이용한 청결 검사}
청결 검사 목표는 세척이 불량한 포일을 찾아서 재세척을 진행하기 위함이다. 청결하지 않은 포일은 고전압을 걸었을 경우 높은 누설 전류를 가지거나, 잦은 방전이 일어나거나 최악의 경우 단락이 된다. 모두 검출기를 사용할 수 없게 되는 조건이다. \uline{이 검사에서 실패한 포일은 메카로에 재세척을 의뢰해야 한다.}

\uline{청결 검사가 진행되는 포일들 외관 검사를 진행하는 포일들과 공간상으로 분리할 것.} 청결 검사를 진행하는 동안 잦은 재세척이 진행되며 포일이 이동하기 때문에 포일이 섞이기 쉽기 때문이다.

\subsection{\SI{600}{\volt} 검사}
\SI{600}{\volt} 검사에서는 포일에 \SI{600}{\volt}의 전압을 걸어 방전이 일어나는지 확인하고 누설 전류를 측정한다. 이 때 전원 공급 장치의 전류 한계 \SI{10}{\micro\ampere} 이상으로 한다. 전류 한계를 높게 설정하는 이유는 충분한 에너지를 공급하여 포일에 있는 오염이 증발하도록 하기 위함이다. 이 때 충분한 에너지가 공급되지 않으면 타다 남은 오염물이 포일에 달라 붙어서 단락을 일으킬 수 있다. \uline{\mbox{\SI{600}{\volt}} 전압을 걸었을 때, \mbox{\SI{5}{\second}} 안에 방전이 멈추고 \mbox{\SI{60}{\nano\ampere}}(=\mbox{\SI{10}{\giga\ohm}})의 누설전류가 관측되어야 한다.}

\SI{600}{\volt}을 걸었을 때 안정적인 포일은 \uline{\mbox{\SI{10}{\volt}} 간격으로 공급 전압을 높여 가면서 방전이 발생하는지 그리고 안정적인 누설전류가 형성되는지 확인한다.} 마찬가지로 전압을 올린 후 \SI{5}{\second} 내에 방전이 멈춰야 한다. 이를 반복하여 \uline{공급 전압은 \mbox{\SI{620}{\volt}}까지 높인다.} \SI{620}{\volt}의 공급 전압에도 방전이 발생하지 않고 \SI{60}{\nano\ampere} 이하의 누설 전류가 흐르면 합격이다.

만약 \uline{5초 후에도 지속적인 방전이 일어나거나, 단락이 일어나거나, 또는 높은 누설 전류가 흐르면 메카로에 해당 포일의 재세척을 의뢰한다.}

\subsection{QC2-Fast}
QC2-Fast는 CMS에서 정립한 포일 검증법으로 포일의 청결도를 시험하기 위해 고안되었다. QC2-Fast을 수행하기 위해 \SI{10}{\minute}이 소요되어 QC2-Long에 비해 짧은 시간이 필요하기 때문에 ``Fast''라고 부른다.

Megger MIT41/2 절연 시험기를 이용해 \uline{포일에 \mbox{\SI{500}{\volt}}의 전압을 걸은 후 \mbox{\SI{10}{\minute}}에 걸쳐 포일 임피던스와 방전 횟수를 기록}한다. 기록은 준비된 Excel 템플릿을 이용한다. 먼저 실험 환경의 \uline{상대습도와 온도를 기록}한다. 전압을 걸은 후 \uline{\mbox{\SI{30}{\second}}, \mbox{\SI{1}{\minute}}, \mbox{\SI{2}{\minute}}, \mbox{\SI{3}{\minute}}, \dots, \mbox{\SI{10}{\minute}}이 지나는 순간의 임피던스를 기록}한다. \uline{초기 \mbox{\SI{30}{\second}} 이후부터 매 시간 간격 사이에 일어난 방전 횟수를 시간 간격의 뒤 시간에 기록한다.} 즉, \SI{30}{\second}부터 \SI{1}{\minute} 사이의 방전 횟수를 \SI{1}{\minute}에 기록하고, \SI{1}{\minute}부터 \SI{2}{\minute}분 사이의 방전 횟수를 \SI{2}{\minute}분에 기록한다. \uline{최종적인 포일 임피던스가 \mbox{\SI{10}{\giga\ohm}} 이상 그리고 방전 속도는 분당 1회 이하가 되어야 한다.}

표 \ref{tab:example_qc2_fast_result}은 일반적인 QC2-Fast 검증 결과의 예이다. 축전이 진행되면서 저항이 시간에 따라 올라가는 것이 정상적인 현상이다. 방전이 일어나면서 포일에 앉은 오염물이 타고 이로 인해 저항이 갑자기 올라가는 경우도 흔하다.

\begin{table}[htb]
  \centering
  \begin{tabular}{|c|c|c|c|c|c|}
    \hline
    시간 (\SI{}{\minute}) & 전압(\SI{}{\volt}) & 저항 (\SI{}{\giga\ohm}) & 전류 (\SI{}{\nano\ampere}) & 방전 횟수 & 총 방전 횟수\\
    \hline
    0.5 & 500 & 3.11 & 176 & - & 0\\
    1   & 500 & 3.73 & 147 & 3 & 3\\
    2   & 500 & 5.9  & 93  & 3 & 6\\
    3   & 500 & 7.6  & 72  & 1 & 7\\
    4   & 500 & 8    & 68  & 2 & 9\\
    5   & 500 & 8.5  & 64  & 1 & 10\\
    6   & 500 & 9.2  & 59  & 0 & 10\\
    7   & 500 & 9.6  & 57  & 0 & 10\\
    8   & 500 & 10.5 & 52  & 0 & 10\\
    9   & 500 & 10   & 55  & 0 & 10\\
    10  & 500 & 11   & 50  & 0 & 10\\
    \hline
  \end{tabular}
  \caption[일반적인 QC2-Fast 검증 결과의 예]{일반적인 QC2-Fast 검증 결과의 예.}
  \label{tab:example_qc2_fast_result}
\end{table}

\uline{만약 임피던스가 낮거나 방전이 많이 일어나는 경우, 메카로에 해당 포일의 재세척을 의뢰한다.}


\subsection{QC2-Long}
QC2-Long은 검출기 내부와 같이 건조한 환경에서 포일의 장기간 안정성을 시험하기 위해 CMS에서 고안된 검증법이다. 포일의 방전 여부에 따라 다르나, 최소 \SI{7}{\hour} 동안 \SI{600}{\volt}의 고전압을 걸어 주었을 때, 포일이 3회 이하의 방전과 \SI{3}{\nano\ampere} 이하의 누설전류를 가져야 한다. \uline{포일이 QC2-Long까지 통과한다면 해당 포일의 품질은 검증된 것으로 간주한다.}

장기간의 걸친 검사를 진행하기 위해 QC2-Long은 컴퓨터가 HV모듈의 제어, 기록을 담당한다. 제어와 기록을 위한 SW은 \url{https://github.com/diracyoon/GEM_QC_SW}에서 다운 받을 수 있다. 해당 SW의 컴파일과 실행법은 \url{https://github.com/diracyoon/GEM_QC_SW}의 README 파일을 참조할 것. 사용자의 편의가 기록의 안정성을 위해 SW은 지속적으로 갱신 중이다.\footnote{불편 사항을 발견하면, 필자에게 알려 주길 바람.}

\subsubsection{QC2-Long의 준비}
건조환 환경을 만들기 위해, 포일을 QC2-Long용 건조 질소 박스에 넣는다, SHV 케이블을 연결한다. \uline{포일의 위치 번호와 SHV 케이블 번호를 잘 기록해 두어야 한다.} 습도를 낮추기 시작하면 포일에 접근이 불가능 하기 때문에 SHV 케이블 연결, 번호 기록에 실수가 없도록 한다. 질소 박스를 닫고, 질소를 최대로 흘린다. QC2-Long을 진행할 수 있는, 상대습도 5\%로 떨어질 때까지 수십분 정도의 시간이 걸린다. 상도습도가 5\%에 도달하면 질소의 flow rate을 조절해 낭비를 막도록 한다.

\subsubsection{Preparation\_QC2\_Long}
상대습도가 떨어지기를 기다리며 Preparation\_QC2\_Long을 실행한다. 해당 프로세스는 QC2\_Long을 위해 장시간에 걸쳐 점진적으로 \SI{450}{\volt}에서  \SI{615}{\volt}까지 전압을 높인다. 각 전압 단계에서 전압이 방전 없이  \SI{10}{\minute}간 유지되면, 전압을 높여 다음 단계로 넘어간다. 만약 한 단계에서 3회 초과의 방전이 일어나면, 해당 전압에서 검사가 실패되고 전압을 낮추어 이전 단계로 돌아간다. 만약 한 전압 단계에서 3회 초과의 실패가 일어나면 Preparation\_QC2\_Long은 실패한 것으로 간주되고 프로세스가 비정상 종료된다. 비정상적으로 프로세스가 종료되면 HV 모듈의 해당 채널 전원이 꺼진다. 반면 포일이 Preparation\_QC2\_Long을 성공적으로 통과하여 해당 프로세스가 정상적으로 종료된 경우, \SI{615}{\volt}의 전압이 유지된다.

만약 \uline{잦은 방전으로 Preparation\_QC2\_Long이 비정상 종료된 경우 메카로에 해당 포일의 재세척을 의뢰한다.}

\subsubsection{QC2-Long}
GEM 포일이 Preparation\_QC2\_Long을 통과했고, 질소 박스내 상대 습도가 5\% 이하로 떨어지면, QC2\_Long을 시도할 수 있다. QC2\_Long이 실행되면 600 V와 100 V을 반복하는 스트레스 테스트를 5회 실시 후, 6시간에 걸쳐 600 V의 전압을 포일에 걸어 주고 누설 전류와 방전 속도를 측정한다. 마지막 단계가 QC2\_Long의 핵심적인 단계이다. 이때 \uline{누설전류가 3 nA 이하로 안정적으로 유지되어야 하며, 방전은 3회 이하로 일어나야 한다. 만약 지나치게 크거나 안정적이지 않은 누설 전류가 관찰되거나, 3회 초과의 방전이 관찰되면 해당 포일은 QC2-Long에서 실패한 것이다.} 3회 초과의 방전이 일어날 경우, QC2\_Long은 비정상 종료된다.

\uline{QC2-Long에서 실패한 포일은 메카로에 재세척을 의뢰한다.} 목표치를 약간 상회하는 안정적인 누설 전류가 관측되었을 경우, 세척 과정 중 고압 DI수 살수 과정만 진행하여도 무방하다.

\subsubsection{Monitor}
Monitor SW을 통해 진행 중이거나 완료된 Preparation\_QC2\_Long와 QC2\_Long의 진행 및 결과를 확인할 수 있다. Monitor는 현재 진행 중인 품질 검증 프로세스의 유무를 감시한다. 만약 진행 중인 프로세스가 감지되면, TCanvas GUI에 각 HV 채널 별로 subpad에 그 결과를 표시한다. 프로세스가 완료되면, 정상 또는 비정상 종료에 관계 없이, 그 결과가 Monitor에 의해 Output 폴더에 저장된다.
