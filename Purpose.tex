\section{품질 검증의 필요성 및 목표} 
네 번의 시험 생산과 검증의 결과 메카로의 품질 표준화에 미진한 부분이 있는 것으로 파악되고 있다. 메카로의 대형 GEM 포일의 생산 능력 자체는 CERN과 비교하여 충분히 훌륭한 수준으로 파악되고 있다. 하지만 부분적으로 성능이 떨어지는 포일이 생산되는 경우가 있다. \uline{저품질의 포일이 CERN으로 배송될 경우 Rui 연구실에서 수리 및 세척 과정을 진행되야 하며, 최악의 경우 해당 포일을 파기해야 할 수도 있다. 이 때 상당한 시간적 및 금전적 손해가 발생한다. 이는 CMS Phase-\Romannum{2} upgrade 프로젝트의 원활한 진행을 방해할 것이다. 또한 KCMS에 추가적인 금전적 손해를 강요할 것이다.}

이런 문제의 발생을 최소화하기 위해 포일 배송 전, GEM 포일의 품질을 최대한 검증하여 문제가 있는 포일의 배송을 막아야 한다. 또한 포일포장법을 표준화 하여 포일이 배송 중 오염되거나 파손되는 일을 최대한 줄여야 한다.

그간 가장 문제가 되었던 부분은 세척의 미비이었다. GE1/1 포일 2차 배송분, 9장의 포일이 CMS에서 진행한 품질검증시험에서 실패하였다. 고전압을 견디는 검증시험에에 모든 포일이 단락되었다. 단락이 된 포일을 Rui 연구실에서 재차 세척을 한 후, 메카로 포일은 검증시험을 문제 없이 통과하였다. 따라서 메카로에서 진행한 포일 세척 과정의 미진함이 문제의 원인으로 해석되었다. 이에 해당 문제를 해결하기 위한 여러가지 시도가 있었다. 포일 생산 차원에서는 메카로의 세척 능력 향상을 위한 노력이 진행 중이며, 최근 상당한 진척을 보이고 있다.

가장 큰 문제였던 세척 문제가 해결되어 감에 따라, 그간 미루어 왔던 문제들을 해결하기 위한 노력이 진행 중이다. 전압선이 단선되는 문제, 홀 지름의 균일성 부족 문제 그리고 식각 결합이 메카로에서 해결해야 할 다음 과제이다. 이에 포일 품질검증 차원에서는 이런 문제들에 대해 검증을 보다 집중하여 진행해야 한다.

\uline{검증 차원에서는 CMS에서 정립된 품질검증법을 포함한 다각도의 품질검증을 메카로에서 진행하여, 우수한 포일을 선별하여 배송하는 방법을 채택하였다.} 이 선별과정의 진행은 매우 효과적이었다. 이 선별과정이 도입된 후 진행된 GE1/1 3차 배송분과 GE2/1 M7 1차 배송분의 품질검증이 큰 문제 없이 진행되었다. 이에 KCMS와 메카로는 CMS의 신뢰를 얻을 수 있었다.

이에 메카로에 파견된 KCMS 구성원이 GEM 포일에 대한 엄격한 품질검증을 진행하고, 올바르게 GEM 포일을 포장한다. 이는 메카로가 GEM 포일 생산에만 집중할 수 있도록 배려함을 위함이다. 또한 생산과 품질검증을 분리하여 품질검증의 기준이 인적 요소에 의해 낮아지는 것을 방지하기 위함이다.
