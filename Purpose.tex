\section{품질 검증의 필요성 및 목표} 
네 번의 시험 생산과 검증의 결과 메카로의 품질 표준화에 미진한 부분이 있는 것으로 파악되고 있다. 메카로의 대형 GEM 포일의 생산 능력 자체는 CERN과 비교하여 충분히 훌륭한 수준으로 파악되고 있다. 하지만 부분적으로 성능이 떨어지는 포일이 생산되는 경우가 있다. \uline{저품질의 포일이 CERN으로 배송될 경우 메카로로 포일을 반송하거나, Rui 연구실에서 수리 및 세척 과정을 진행되야 하며, 최악의 경우 조립이 완료된 챔버를 다시 분해하고 해당 포일을 파기해야 할 수도 있다. 이 때 상당한 시간적 및 금전적 손해가 발생한다. 이는 CMS Phase-\Romannum{2} upgrade 프로젝트의 원활한 진행을 방해할 것이다.}

이런 문제의 발생을 최소화하기 위해 포일 배송 전, GEM 포일의 품질을 최대한 검증하여 문제가 있는 포일의 배송을 막아야 한다. 또한 포일포장법을 표준화 하여 포일이 배송 중 오염되거나 파손되는 일을 최대한 줄여야 한다.

그간 가장 문제가 되었던 부분은 포일의 최종 세척 조건의 불안정성이었다. GE1/1 포일 2차 배송분, 9장의 포일이 CMS에서 진행한 품질검증시험에서 실패하였다. 고전압을 견디는 검증시험에에 모든 포일이 단락되었다. 단락이 된 포일을 Rui 연구실에서 재차 세척을 한 후, 메카로 포일은 검증시험을 문제 없이 통과하였다. 따라서 메카로에서 진행한 포일 세척 과정의 미진함이 문제의 원인으로 지목되었다.

최근 발생했던 가장 큰 문제는 GE2/1 M7 포일 1차 배송분에서 발생한 홀 지름의 문제와 그로 인한 이득 하락 문제\footnote{\url{https://indico.cern.ch/event/816612/contributions/3408987/attachments/1835117/3006275/GEM_Phase2_Upgrade_Coordination_Meeting_20190426.pdf}}였다. 이 문제의 이유는 구리 홀의 지름이 스펙보다 길었고, PI 홀의 지름은 스펙보다 짧았기 때문이다. 식각에 필요한 시간을 보정하는 과정에서 사용된 광학계에서 발생한 문제가 원인으로 생각된다. 해당 챔버는 분해되어 사용된 포일을 폐기하였다.     

이외에도 전압선이 단선되는 문제, 홀 지름의 균일성 부족 문제 그리고 식각 결합이 메카로에서 해결해야 할 과제이다.

\uline{품질검증 차원에서는 CMS에서 정립된 품질검증법을 포함한 다각도의 품질검증을 메카로에서 진행하여, 우수한 포일을 선별하여 배송하는 방법을 채택하였다.} 이에 메카로에 파견된 KCMS 구성원이 GEM 포일에 대한 엄격한 품질검증을 진행하고, 올바르게 GEM 포일을 포장하여야 한다.

포장된 포일은 배송 업체를 통해 CMS로 배송된다. CMS에 도착한 포일을 재차 검수하는 과정도 KCMS의 책임이다. 포일이 검수를 통과하지 못할 경우, 해당 포일을 메카로로 반송하거나 Rui 연구실에서 수리해야 하는데, 이 과정도 KCMS의 책임이다.
